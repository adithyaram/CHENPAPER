\documentclass[aps,secnumarabic,nobalancelastpage,amssymb,
nofootinbib,nobibnotes,amsmath,prl,longbibliography,12pt]{revtex4-1}
%\documentclass[secnumarabic,floatfix,nofootinbib,
%tightenlines,nobibnotes,aps,prl,,notitlepage12pt]{revtex4}
\usepackage[left=2.5cm,right=2.5cm,top=2.5cm,bottom=1.5cm,letterpaper]{geometry}
\usepackage{graphics}      
\usepackage{longtable}     
\usepackage{float}
\restylefloat{table}
\usepackage{siunitx}
\usepackage{natbib}
\usepackage[english]{babel}
\usepackage{placeins}
\usepackage{hyperref}
\hypersetup{
  colorlinks=true,
  linkcolor=black,
  citecolor=black,
  filecolor=black,
  urlcolor=black,}

\begin{document}

\begin{titlepage}

\title{Drying of Discotic Suspensions}
\author{Adithyaram Narayan}
\email[Electronic address: ]{adinar@tamu.edu}


\affiliation{Mary Kay O'Connor Process Safety Center, Artie McFerrin Department of Chemical Engineering, Texas A\&M University, College Station, Texas 77843-3122, USA}


\begin{abstract}
The self-assembly of particles via evaporation of the solvent is a growing and vibrant field in the area of materials science and engineering. The self and directed assembly of colloidal particles are of use 
\end{abstract}

\maketitle


\end{titlepage}

\section{Introduction}

\subsection{Self Assembly}
The advent and increasing usage of nanoparticles in everyday process has gained interest and a new thrust in the engineering of materials. The one feature of nano-particles are their ability to assemble spontaneously into large structures due to non-bonding forces at that become important at these length scales.\\ A general definition of self-assembly is the spontaneous organization of materials to mesoscopic structures via non-covalent interactions(hydrogen bonding, Van-Der Waals forces, $\pi-\pi$ interactions, electrostatic forces)\cite{SelfADMA}. 

\subsection{The Coffee-Ring Effect}
The coffee ring effect in non-equilibrium condensed matter physics is the phenomenon by which the suspended particles in an evaporating drop are deposited in a ring like fashion near the air-drop interface. The explanation to this effect was first proposed by Robert Deegan\cite{deegan}\cite{deegan1997capillary} where he showed that when a contact line is pinned, the evaporating fluid flows radially forcing the particles to migrate towards the drop interface thereby depositing as a ring-like structure. While the coffee-ring effect is observed in small drops, the effect is formed by complex interplay of hydrodynamics, solvent interactions and material properties like physical properties of drying substrate, polydispersity of the suspended particles, particle-particle interactions\cite{C2NR30286A} etc.\\ 

The coffee-ring effect is an important macro and micro effect, tuning this physical phenomenon is of importance and can lead to applications such as printing\cite{Inkjet} for manufacturing electronic panels, intelligent self-assembled   structures\cite{C4SM01784F} and other interesting physical phenomenon\cite{C2SM27089G},\cite{Stoneparticle},\cite{PhysRevE.53.1994}. Due to the robustness of this effect, it has been used to pattern solutes on a substrate or to form interesting patterns as is manifested in proteins\cite{Proteinsensor}, drying of layered double hydroxide platelets\cite{Zhang201311}, complex self-assembly of nano-crystals\cite{MinPRL}, clay films\cite{wang2013self}, sub-micrometer rings\cite{submicroange}, complex fluids like blood\cite{Sobac201434} and even three dimensional assemblies\cite{Choi}. Studies have also focused on computational studies\cite{compunan}\cite{LBsim}.This effect is well studied due the simplicity of performing the experiment . These complex assemblies can be prepared by drying a drop on a suitable substrate. The control and suppression of the coffee-ring effect is of importance to form homogeneous thin films and coatings.\cite{semiconductorgrowth}\\

In a recent discovery by Yunker and co-workers\cite{yodhnat} the aspect ratio of the suspended particles was found to be important in the manifestation of the coffee-ring". Indeed in different works of theory and experiments, scientists have attempted to control the coffee-ring effects. The coffee-ring effect can be hindrance to complex self-assembly, printing/lithography and other macro/micro level tasks where uniformity is of importance. This discovery of tuning the coffee ring by using anisotropic particles opens doors to applications. Interestingly, when the marangoni flow effects are amplified by changing the surface tension by either using surfactant\cite{YodhSDS} or by using a different solvent which enhances the marangoni flow\cite{hu2006marangoni} the coffee-ring effect manifests due to re circulation of the solvent in the drop back from the interface to the center of the drop. The self-assembly of other types of highly anisotropic shapes such as ZnO-nanorods\cite{C4SM00887A} proceeded via mechanism that is different from the agreed upon model of the coffee-ring effect.\\

The work by Yunker et al.\cite{yodhnat} examined the effect of the coffee ring by using spherical and ellipsoidal particles of $\xi$=3.5 and discovered that the ellipsoidal structures were successful in avoiding the coffee rings.



\section{Conclusions}


\bibliography{CHEN689}


\end{document}

